% \section{Introduction}
% \label{sec:intro}
In our previous work~\cite{casper}, we proposed ``Casper,''
a process-based asynchronous progress solution for MPI on multi- and
many-core architectures. An alternative to traditional thread- and
interrupt-based models, Casper provides a distinct set of benefits
for applications, which we believe are more appropriate for large
many-core architectures. The philosophy of the Casper architecture
is centered on the notion that since the number of cores in the system
is growing rapidly, there can be always a few idle cores during the
execution, thus dedicating them for helping with asynchronous progress
might be better than using an interrupt-based model. Similarly, the use
of processes rather than threads allows Casper to control the amount
of sharing, thus reducing thread-safety overheads associated with
multithreaded models, as well as to control the number of cores being
utilized for asynchronous progress.

The central idea of Casper is the ability of processes to share
memory by mapping a common memory object into their address spaces
by using the MPI-3 shared memory windows interface.  Specifically, Casper
keeps aside a small user-specified number of cores on a multi- or
many-core environment as ``ghost processes.''  When the application
process tries to allocate a remotely accessible memory window, Casper
intercepts the call and maps such memory into the ghost processes'
address space.  Casper then intercepts all RMA operations to the user
processes on this window and redirects them to the ghost processes
instead.

We have evaluated Casper by using various micro-benchmarks and a real
quantum chemistry application, NWChem, and shown significant performance
improvement in our previous work. However, the performance is still not
ideal. Although we have achieved signification improvement in some
computation-intensive phase such as the time-consuming (T) portion in
the ``gold standard'' CCSD(T) simulation~\cite{casper}\cite{casper-scaling},
we also observe performance degradation in some communication-intensive
phases such as CCSD iteration. This is because, in communication-intensive
phases, the asynchronous progress may not be necessary since the target
processes can frequently make MPI calls and thus handle the operations
issued to them as targets by themselves. Furthermore, when the amount of RMA
operations becomes large, the operation redirection in Casper can even
result in over workload issue. That is, a large amount of operations
were originally issued to different user target processes, but are
redirected to a few ghost processes in Casper. Such situation can result in
communication performance degradation if the performance benefit from
asynchronous progress is less than the performance loss in the operation
overloading on ghost processes.

In this paper, we propose an adaptation mechanism for Casper, providing
the capability to dynamically adapt the asynchronous progress
for multiple execution phases performing varying communication characteristics.
The adaptation approach allows application to dynamically redirect all
RMA communication to the ghost processes for computation-intensive phase
that requiring efficient asynchronous progress, with avoiding inefficient
redirection for communication-intensive phase.
We note that, the cores dedicated to background ghost processes are always
kept aside from the application even when the asynchronous progress is disabled.

\mynote{Challenge.}

We summarize the contribution of this paper as follows:
\begin{itemize}
\item Deep performance analysis of the widely used CCSD and CCSD(T)
simulations in NWChem for two problem systems: the water system and
the acenes molecule system.

\item Two dynamic adaptation approaches for Casper asynchronous
progress: a user-guided approach and a transparent self-profiling
based approach.

\item Comprehensive comparison between the adaptation approaches in
the CCSD(T) simulation.
\end{itemize}

% The rest of the paper is organized as follows: Section~\ref{sec:des-basic}
% first gives an overview of the basic design of Casper. Then
% Section~\ref{sec:eva-nwchem} presents the deep analysis of the performance
% characteristics of NWChem and discusses the limitation of static asynchronous
% progress. Section~\ref{sec:des-adpt} introduces the detailed design of two
% dynamic adaptation approaches proposed in this paper, and followed by the
% comprehensive performance analysis in microbenchmarks and NWChem in
% Section~\ref{sec:eva-micro} and ~\ref{sec:eva-nwchem-adpt} respectively.
% The related researches are introduced in Section~\ref{sec:related}, and
% finally this paper is concluded in Section~\ref{sec:concl} with a discussion
% of future work.
