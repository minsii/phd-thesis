%%%%%%%%%%%%%%%%%%%%%%%%%%%%%%%%%%%%%%%%%%%%%%%%%%%%%%%%%%%%%%%%%%%%%
\chapter{Introduction}\label{sec:intro}
%%%%%%%%%%%%%%%%%%%%%%%%%%%%%%%%%%%%%%%%%%%%%%%%%%%%%%%%%%%%%%%%%%%%%

\section{Heterogeneous Many-Core Architectures}
The previous decade, which marked the increase of compute power from
terascale (ASCI Red supercomputer, deployed at Sandia in 1996~\cite{ascired}) to
petascale (Roadrunner supercomputer, deployed at Los Alamos in 2008~\cite{roadrunner}),
passed without a significant paradigm shift: most of the performance
improvement came from making each processor faster.
But this improvement was at the cost
of using more power per processor; and, indeed, the arrival of
petaflop supercomputers coincided with processors hitting the power
wall whereby any additional increase in the power usage of a processor would
result in the processor's components melting or becoming extremely
unreliable.  Consequently, each processor can no longer become
significantly faster.
Another three orders of magnitude increase
in performance will require at least three orders of magnitude
increase in the processing components.  This can come from
more nodes, processor cores, or threads, all of which add significant
complexity to the supercomputer.
We are already beyond the traditional work-distribution paradigm of
``equal work means equal time.''  That is, on traditional
supercomputers, if we distribute the work across processors such that
each processor gets an equal amount of work, then they take an
equal amount of time to compute.  Such fundamental assumptions are
already falling apart on today's supercomputers and will continue to
collapse in future supercomputers, primarily because of the complexity
of the processor and memory architectures involved.


\section{Complex and Irregular Scientific Applications}

The issues that arise with such a paradigm shift are not just
theoretical; indeed, they are already being seen in many application
domains.  One example is in quantum chemistry simulation.
Specifically, with the widely used software NWChem~\cite{nwchem}
(developed by PNNL), the overhead of irregularity in data
communication across processors and the resulting imbalance in the
computational load dominate the cost of current simulations (e.g., the
largest simulations of NWChem spend more than 50\% time idling).  As
researchers move to much larger simulations, this imbalance is only
going to get worse.  Another example is in bioinformatics.  The Kiki
genome assembly application~\cite{kiki}, developed by Argonne, is used
in metagenomics to understand complex environments containing many
different microbiomes (e.g., soil).  Because of restrictions in
current genome sequencing hardware, the problem tends to result in a
large number of DNA fragments that need to be assembled back into
their original complete DNA chains.  This involves enormous irregular
data movement over petabytes of data and requires several days or even
months of computation.  For instance, the largest simulation done to
date was at U. Chicago, where a 2.3-terabyte sample was assembled on a
supercomputer with 18,000 cores---this simulation took 4 days to
complete and spent 99.9\% of its time idling, because of imbalance
between processing units.

\section{Motivation}

Since multicore processors have become the most common processor
architectures today, the only way to improve the performance for
high-end processors is to add more threads and cores. Many-core
architectures, such as Intel Xeon Phi and Blue Gene/Q, provide such a
massively parallel environment with dozens of cores and hundreds of
hardware threads. More and more scientific application programmers
have begun investigating ways to utilize such architecture for scaling
application performance.

Although many-core architecture can provide the enormous power of
parallel computing, application performance may still be restricted in
various ways. Two characteristics of such hardware have to be taken
into account. First, cores are designed to be simple and low frequency
for a better performance-to-energy ratio; thus, execution on a single
core could result in extreme performance degradation. Second, the
number of cores is increasing at a faster rate than other on-chip
resources (e.g., memory), potentially resulting in scalability
bottlenecks.

To better utilize such hardware resources, application programmers
have studied several approaches that provide better parallelism and
resource sharing for different scientific applications. Many of those
approaches, however, still face communication problems that may result
in performance degradation. This doctoral research aims to exploit the
capabilities of many-core architectures on the widely used
message-passing model and propose techniques for solving existing
communication problems.


\section{Contribution}

In this thesis, we focus on optimizing the
communication of the two most popular programming models used in
modern applications: a hybrid MPI+threads model and an MPI one-sided
communication model.

\vspace{0.2ex}
\noindent\textbf{Communication optimization in hybrid MPI+threads
  model}.  An increasing number of applications are looking at hybrid
programming models, frequently called ``MPI+threads,'' to allow
resources to be shared between different cores on the node. A common
mode of operation in such hybrid models involves using multiple
threads to parallelize computation within the node, but using only one
thread to issue MPI communication. Although such a mode achieves
significant improvement in floating-point computing by massive
parallelism, it also means that most of the threads are idle during
MPI calls, a situation that translates to underutilized hardware
cores. Furthermore, since MPI communication performs only on a single
lightweight core, this mode may even result in performance
degradation.

To resolve the problems in the MPI communication of hybrid model, we
present MT-MPI~\cite{mtmpi}, an internally multithreaded MPI that
transparently coordinates with the threading runtime system to share
idle threads with the application in order to parallelize MPI internal
processing such as derived datatype communication, shared-memory
communication, and network I/O operations.

\vspace{0.2ex}
\noindent\textbf{Optimization for MPI one-sided communication}.  For
applications with large memory requirements, developers start sharing
memory resources across nodes through a global shared address space
implemented by employing MPI one-sided
communication~\cite{dinan12:armci_mpi}.  The MPI-2 and MPI-3
standards~\cite{mpi30-report} introduced one-sided communication (also
known as remote memory access or RMA), which allows one process to
specify all communication parameters for both sender and receiver.
Thus a process can access memory regions of other processes in the
system without the target process explicitly needing to receive or
process the message. Although such communication semantics can
asynchronously handle communication progress and hence hide
communication overheads from computation, it is not truly asynchronous
in most MPI implementations. For example, although contiguous PUT/GET
MPI RMA communication can be implemented in hardware on RDMA-supported
networks such as InfiniBand, thus allowing the hardware to
asynchronously handle its progress semantics, complex RMA
communication such as an accumulate operation on a 3D subarray must
still be done in software within the MPI implementation.
Consequently, the operation cannot complete at the target process
without explicitly making MPI progress and thus may cause arbitrarily
long delays if the target process is busy computing outside the MPI
stack.

To resolve the problem of asynchronous progress, we propose
Casper~\cite{casper}, a process-based asynchronous progress model for
MPI one-sided communication on multicore and many-core architectures,
that keeps aside a small user-specified number of cores as background
``ghost processes'' to help asynchronous progress. The philosophy of
Casper is centered on the notion that since the number of available
cores in modern many-core systems is increasing rapidly, some of the
cores might not always be busy with computation and can be dedicated
to helping with asynchronous progress.

In summary, this Ph.D. thesis aims to enable highly efficient message
passing on many-core architectures for various kinds of scientific
applications. Two techniques are proposed to address different
communication issues existing in modern applications.  In
Sections~\ref{sec:mtmpi} and \ref{sec:casper}, we separately present
their current state and sketch the next steps.

\subsection{Outline}
