近年、マルチコアプロセッサが広く普及されるが、消費電力と発熱問題などの理由により、
従来の動作周波数の向上によるプロセッサの処理性能向上は困難になっており、比較的に効率
的な方法はコア数を追加して更なる並列化しかない。インテル社のXeon Phiや、IBM社のBlue 
Gene/Qなどのメニーコアアーキテクチャーは我々に数十コア・数百ハードウェアスレッドの
ような大規模並列環境を提供する。このようなアーキテクチャーのハードウェア特徴をまとめると、
一つは膨大なコア数により大規模並列計算環境が提供されることで、もう一つは省エネルギーのために
各プロセッサコアの動作周波数が低く抑えされることである。それらの特徴に従い、メニーコアの
性能を発揮するには、なるべくアプロケーション全体を大規模並列化すべきことがわかるが、逆に並列
コア数が不足の場合、性能が悪化することも注意しないと行けない。一方、アプリケーション側
では、ソーシャルネットワークやバイオインフォマティクス分野などにおけるデータ集約型
アプリケーションが大幅に増加されている。ステンシル計算などの規則的な計算パターンを持つ
アプケーションと違い、これらのアプリケーションは常にデータ駆動型の計算方式を採用してトポロジー
が不規則でいつも動的に変更され、今までの規則的な通信パターンによく使われ送受信・集約通信モ
デルでは実現しにくい。


本論文は、科学計算に広く用いられるメッセージパッシング通信モデルを対象としてメニーコアの特徴を
活用して通信性能を最大限に発揮し、あらゆる規則・不規則的な科学計算アプリケーションに最適な通信
手法を提案して全体性能の向上に貢献する。本論文は以下二つの課題から議論を展開する。一つ目は、計算
部分も通信部分も含めてアプリケーションの全体を通してメニーコア上の大規模並列資源を最大限に活用
することである。二つ目は、メニーコアの特徴を活用して、新しい不規則計算に最適な通信最適化手法を
提案することである。

まず、一つ目の課題に関して、本論文はメニーコア上で実行する様々なプログラミングモデルを深く解析
して通信部分の不足を発見し、それぞれに効率的な最適化手法を提案する。

メモリやその他のシステム資源に比べて、プロセッサー数が常により高速に増加されている等の理由で、
アプリケーションプログラマーがますますプロセスとスレッドを混在させるハイブリッドプログラミ
ングモデル、いわゆる「MPI+X」モデルに注目する。このようなモデルでは、同一プロセスに所属する複数
スレッドがノード内の資源を共有することが可能となり、スレッドモデルの主役であるOpenMPと分散メモ
リシステム上で通信機能を担当するMPIライブラリの組み合わせが主流となる。現在一番使われているハイブ
リッドMPI+OpenMP手法は、複数のOpenMPスレッドが計算を並列化してその中の一つがMPI通信を行うこと
である。このようなパターンでは、浮動小数点計算を大規模並列化することにより計算部分の性能向上が
達成されたが、通信部分はまだ従来通り一つの軽量コアだけが担当し、全体性能障害の原因ともなる。本論
文はこの問題に対して、アイドル状態になったユーザアプリケーションが作成したスレッドを再利用し
てMPI内部通信作業を並列化する手法を提案する。この手法を実現するには、まずMPIにスレッド情報
を共有できるようにOpenMPランタイム側を変更してスレッド情報を参照できるように変更した。次はMPI実
装側を変更して、スレッド情報により様々な内部通信作業に並列アルゴリズムを適用する。例として、本論
文はユーザ定義データ型通信、共有メモリ通信とネットワークI/O作業を最適化し評価する。

一方、ハイブリッドMPI+スレッド型プログラミングの増加にも関わらず、一MPIプロセスが一コア上で走る
MPIプロセスだけのプログラミングモデルはまだ一部のアプリケーションに使われている。このようなモデル
では、大量プロセス間で計算資源の割合などの問題だけでなく、性能の面でも不利点が色々ある。例えば、
一MPIプロセスがメッセージを待つ時、メッセージが到着するまでその計算コアがアイドルになり性能を発
揮できなくなる。本論文は、一コア上でOSプロセスを複数スケジュールできるユーザレベルプロセスのア
プローチに基づき、大量のMPIプロセスを一コア上で実行して負荷分散やチェックポイントの軽量化などの
面から改良手法を提案する。

二つ目の課題に関しては、本論文はメニーコア上で不規則計算の通信モデルに対する最適化に着目する。
不規則計算型アプリケーションでは、データは常にグラフなどの疎な構造体であり、通信・計算パターンも
いつも不規則で動的に変更されるため、伝統の規則通信に用いた送受信・集団通信手法を適用することが
困難になり、このような科学計算に最適する片方向通信を使うアプリケーションが増えてくる。MPI-2規格
から定義された片方向通信、いわゆる「Remote Memory Access、RMA」は、通信の相手の状態に無関係
に他のプロセスのデータにアクセスできる通信モデルである。このようなモデルは不規則計算の通信パターン
に最適だけでなく、特にInfiniBandやCray Ariesインタコネクトなどのハードウェア片方向通信をサポート
するネットワーク上で、より効率的、自然的な通信・計算間のオーバーラップを実現する可能性も高い。
しかしながら、MPI規則ではこのような通信モデルが必ず非同期であることは保証されていないし、RMA通信を
完成するためにリモート側プロセスがMPIを呼び出さないといけないと大分のMPI実装はまだ要求する。この
課題に対して、研究者たちは常に非同期処理のアプローチを注目する。本論文は、メニーコアの特徴を活用し、
プロセスレベルのMPI非同期通信専用ヘルプコアの最適化手法を提案する。この手法は、MPI-3共有メモリ
Window機能とPMPIリダイレクト手段を用いて、ハードウェアサポートのRMA通信を影響せずにソフトウェア
介入のRMA通信を非同期させる。更に、PMPIリダイレクトを利用したMPI外部実装方式はあらゆるMPI実装にも
容易にサポートできる利点もある。初期段階の評価により、この手法は優れた通信・計算オーバーラップとスケ
ーラビリティを達成する。

共有メモリWindowを用いたアプローチは、片方向通信モデルの非同期通信専用ヘルプコアをMPI外部で実現
できたが、伝統的な送受信通信や集団通信に対してヘルプコアの実装はまだ困難である。すべての通信モデルに
も適用できる汎用的なアプローチを実現するために、本論文は特殊なメモリマッピング技術を用いて上記実装を
拡張すると予定する。
