%%%%%%%%%%%%%%%%%%%%%%%%%%%%%%%%%%%%%%%%%%%%%%%%%%%%
\section{Active-target RMA}\label{sec:active}
%%%%%%%%%%%%%%%%%%%%%%%%%%%%%%%%%%%%%%%%%%%%%%%%%%%%
After clarifying the design for passive-target locks, we
expand it to active-target mode in this section. Since the active-target mode does a
similar synchronization as the passive mode except that the target
side also explicitly performs synchronization, there is no semantic
issue if we internally translate such active synchronization calls
to passive locks. The active-target synchronization call can be
\textit{fence} or \textit{PSCW}. We clarify each separately in
the following paragraphs.

\parahead{MPI\_Win\_fence} This collective synchronization call supports
a simple synchronization pattern that allows a process to access
windows at all processes in the group of windows, and the local window
of a process can be also accessed by all processes in the group
in between a pair of fence calls. To be specific, a fence call completes
an epoch if it was preceded by another fence (usually the first
fence), and starts an epoch if it is followed by another fence
(usually the second fence). The implementation of such call always
consists of the permission synchronization among all the processes
of that window followed by a collective call as a barrier.

Definitely, such processing can be translated as a
\textit{unlock\_all-barrier-lock\_all} behavior inside Casper,
whose correctness is already guaranteed as clarified in the passive-target
mode. Since we do not need any permission synchronization between
active-target and passive-target epochs like that between \textit{lock}
and \textit{lock\_all}, we simply use a separate internal window
containing all the user processes and ghost processes for fence
calls. Such design also eliminates any potential conflict in active
and passive-target mixed programs.

It is noted that users can specify \textit{MPI\_MODE\_NOPRECEDE} and
\textit{MPI\_MODE\_NOSUCCEED} for the first fence and the last one
respectively, in order to eliminate unnecessary internal
synchronizations. Obviously, such optimization can be also used in
Casper lock-based fence. Specifically, a fence with
\textit{noprecede} does not complete any sequence of locally issued
RMA calls, and thus the \textit{unlock\_all-barrier} can be
eliminated. On the other hand, a fence with \textit{nosucceed} does not start any
locally issued RMA calls, which means that the \textit{lock\_all}
can be avoided.

\parahead{PSCW} allows synchronization between pairs of communicating
processes. \textit{MPI\_Win\_start} and \textit{MPI\_Win\_complete}
are the calls for starting and terminating an epoch to access a group
on the origin process, while \textit{MPI\_Win\_post} and \textit{MPI\_Win\_wait}
are used for exposing and closing an epoch on the target process.
Similarly, such synchronization calls can be also translated to
passive lock calls: \textit{start} and \textit{complete} are
internally translated as \textit{lock} and \textit{unlock}
respectively. The \textit{post} call on the target process is easily
translated to an empty call, and that target process waits for the
completion of the origin side in a \textit{wait} call which can be
implemented as a simple two-sided send\slash receive synchronization
or a one-sided version in that the target waits for a counter to be updated
to zero by the origin processes using RMA calls.


\textcolor{red}{[NOTE] Active mode is changed to new global-window design.}

\textcolor{red}{[NOTE] Need discuss why we need win\_sync in active mode by default.}
\textcolor{red}{[NOTE] Need discuss assert optimizations, especially win\_sync.}
In Fence :
if (no store) drop (still need win\_sync to avoid load instruction
reordering.);
if (no put) drop ? (x=0;fence;remote get(x);fence;x=1; can get return 1 ? or get handler should do mb after read?)
if (no precede) no flushall;
if (no succeed) drop (still need barrier to make sure any RMA to my
window should be done, flushall only ensure locally issued RMA);

In Post-Start:
if (no check) no send-recv in post-start.
if (no store) drop (still need win\_sync to avoid load instruction
reordering.);
if (no put) drop ? (x=0;post;remote get(x);wait;x=1; can get return 1 ? or get handler should do mb after read?)

\textcolor{red}{
[NOTE] Current Casper does not pass non\_contiguous to win\_allocate\_shared.
Non contiguous would have been a good performance enhancement to pass, but is not strictly necessary.  In fact we can argue in the paper that it'll increase the buffer usage in Casper, which is why we didn't use it.}