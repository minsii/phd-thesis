%%%%%%%%%%%%%%%%%%%%%%%%%%%%%%%%%%%%%%%%%%%%%%%%%%%%%%%%%%%%%%%%%%%%%%%%%%%%%
\section{Many-Core Architectures}\label{sec:back-manycore}
%%%%%%%%%%%%%%%%%%%%%%%%%%%%%%%%%%%%%%%%%%%%%%%%%%%%%%%%%%%%%%%%%%%%%%%%%%%%%

The Intel Xeon Phi architecture features a large number of CPU cores
inside a single chip.  The Xeon Phi cards run their own Linux-based
operating system and can launch full operating system processes.  In
the native mode, system calls that cannot be handled directly on the
Xeon Phi card are transparently forwarded to the host processor, which
executes them and sends the result back to the issuing process.
Although these devices also offer the possibility of running in
\emph{offload mode}, following a GPU-like approach, this mode is not
considered in our research because it does not allow the coprocessors
to run hybrid MPI + OpenMP applications.

When MPI processes are launched on a combination of multiple nodes and
adapters, these processes internally communicate with each other using
a number of mechanisms.  Processes on the same Xeon Phi card
communicate with each other using shared memory.  Processes on the
same node communicate using the PCIe peer-to-peer capabilities.  When
communicating outside the node, for some networks such as InfiniBand,
communication is performed directly without host intervention through
the PCIe root complex.

The first generation of the product released to the public, code-named
Knights Corner~\cite{knc}, features a minimum of 60 simple cores each
capable of 4 hardware threads, providing a total of 240 hardware
threads per coprocessor.  The card is equipped with 8~GB of GDDR5 RAM.
One difference between this architecture and GPU architectures is the
fully private and coherent cache provided to each processing unit:
32~KB instruction + 32~KB data L1, and 512~KB L2 (unified), offering
high data bandwidth.  Further details on the Intel Xeon Phi
architecture can be found in~\cite{mic,knc}.


