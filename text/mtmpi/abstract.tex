\begin{abstract}

Many-core architectures, such as the Intel Xeon Phi, provide dozens of
cores and hundreds of hardware threads.  To utilize such
architectures, application programmers are increasingly looking at
hybrid programming models, where multiple threads interact with the
MPI library (frequently called ``MPI+X'' models).  A common mode of
operation for such applications uses multiple threads to
parallelize the computation, while one of the threads also issues MPI
operations (i.e., MPI \texttt{FUNNELED} or \texttt{SERIALIZED}
thread-safety mode).  In MPI+OpenMP applications, this is achieved,
for example, by placing MPI calls in OpenMP critical sections or
outside the OpenMP parallel regions.  However, such a model often
means that the OpenMP threads are active only during the parallel
computation phase and idle during the MPI calls, resulting in wasted
computational resources.  In this paper, we present MT-MPI, an
internally multithreaded MPI implementation that transparently
coordinates with the threading runtime system to share idle threads
with the application.  It is designed in the context of OpenMP and
requires modifications to both the MPI implementation and the OpenMP
runtime in order to share appropriate information between them.  We
demonstrate the benefit of such internal parallelism for various
aspects of MPI processing, including derived datatype communication,
shared-memory communication, and network I/O operations.

\end{abstract}

%% \keywords Multithreading; MPI; OpenMP; hybrid; Xeon-Phi; Many-core;
%% Datatype; RMA; Intra-node; InfiniBand
