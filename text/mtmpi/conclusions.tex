%%%%%%%%%%%%%%%%%%%%%%%%%%%%%%%%%%%%%%%%%%%%%%%%%%%%%%%%%%%%%%%%%%%%%%%%%%%%%%
\section{Conclusions}\label{sec:concl}
%%%%%%%%%%%%%%%%%%%%%%%%%%%%%%%%%%%%%%%%%%%%%%%%%%%%%%%%%%%%%%%%%%%%%%%%%%%%%%

In this paper we analyzed the potential benefits of employing idle
hardware threads to accelerate MPI processing in hybrid MPI+OpenMP
applications on massively parallel many-core architectures.  To
this end, we modified two widely deployed
implementations of these models: the Intel OpenMP runtime and MPICH
MPI implementation. We described various optimizations in different
parts of MPI implementation, including derived datatype processing,
shared-memory communication, and IB network operations.  Our
experimental evaluation, based on several micro- and macrokernels,
demonstrates considerable performance benefits.
