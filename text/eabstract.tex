Since multicore processors have become the most common processor architectures today, the only way to 
improve the performance for high end processors is to add more threads 
and cores. Many core architecture, such as Intel Xeon Phi and Blue Gene/Q, 
provides us such a massively parallel environment with dozens of cores and
hundreds of hardware threads. we highlight their specific hardware features 
that, first, large amount of cores provide massively parallel computing 
environment; and second, chips are embedded with low frequency and low power
cores for better performance to energy ratio. 
According to above features, we understand that utilizing such large amount 
of cores provides us efficient parallelism, 
however, if only one or a few cores are utilized we may even lose performance. 
Scientific application programmers are increasingly looking at the ways to 
utilize such large amount of light-weight cores for various programming 
models in a combination of computation and communication. 

From the aspect of computation, with the number of processing cores increasing at 
a faster rate than other resources in the system, more and more applications 
start focusing on hybrid programming models comprising a mixture of processes 
and threads, that allow resources on a node to be shared between the different 
threads of a process. The most prominent of the 
threading models used in scientific computing today is OpenMP. In such hybrid models, 
multiple OpenMP threads parallelize the computation, while one or more threads 
utilize a distributed-memory programming system, such as MPI, for their data 
communication. On the other hand, despite the increasing trends in hybrid MPI+threads
applications, a large number of applications still function in an MPI-only 
mode, where only single thread exists in each MPI process and an MPI process 
is launched for each available core on the system.

From the view of communication, message passing is the prominent of communication
models over distributed-memory systems. MPI is the most widely used standard for 
writing such communication model. An MPI progress can be synchronous or asynchronous. 
In traditional two-sided send\slash receive communication mode, a synchronous MPI progress, such as 
blocking send, can be finished only after the message has been received by receiver;
non-blocking operations are asynchronous progress, in which a send 
operation can be finished immediately without waiting for message reception. 
Moreover, the MPI-2 standard introduced one-sided communication semantics (also known as RMA), 
that allows one process to specify all communication parameters, for both the 
sending side and the receiving side. Thus, a process can access memory regions 
of other processes in the system without the target process explicitly needing 
to receive or process the message. RMA has the potential to be fully asynchronous, 
particularly on networks that support one-sided communication natively, 
such as InfiniBand, Fujitsu Tofu interconnect or Cray Aries interconnect.

Different applications benefit from different combination of computation and 
communication models. However, it is not easy to efficiently execute these models
on massively parallel many core environment, performance may be degraded in various
ways. This thesis focuses on exploiting the capabilities of many core architectures
on widely used message passing model, in order to address these issues existing 
in different programming models and consequently contribute efficient communication 
approaches for various kinds of applications. 

Firstly, in hybrid MPI+threads applications, a common mode of operation for such 
applications involves using multiple threads to parallelize the computation, 
while one of the threads issues MPI operations. Although such mode extremely 
improves floating point performance for computation of applications by massive 
parallelism, it also means that most of the threads are idle during MPI calls, 
which translate to underutilized hardware cores. Furthermore, since only single 
light-weight core is contributing to communication, it may result in even 
performance degradation. The first contribution of this thesis is an internally multithreaded 
MPI that transparently coordinates with the threading runtime system to \textit{share
idle threads with the application in order to parallelize MPI internal 
processing} such as derived datatype communication, shared-memory communication, 
and network I/O operations. 
% It is designed in the context of OpenMP and requires 
% modifications to both the MPI implementation and the OpenMP runtime in 
% order to share appropriate information between them. We demonstrate the 
% benefit of such internal parallelism for various aspects of MPI processing, 
% including derived datatype communication, shared-memory communication, 
% and network I/O operations.

Secondly, in MPI-only applications, such a mode not only suffers from the partitioning 
of resources across processes, but also from loss of performance in a number 
of ways. For example, if an MPI process is waiting for a message to arrive, 
the core on which it is scheduled is idle and underutilized. As the second 
contribution of this thesis, we plan to utilize the concept of user-level 
processes, a way to provide multiple co-scheduled ``OS processes'' on a 
single core, so that \textit{a large number of MPI processes are able 
to be executed on a single core} and achieve better load balancing and 
light-weight checkpoint migration.


Finally, with regard to asynchronous communication, however, the MPI standard does 
not guarantee that such communication is truly asynchronous. Even in one-sided 
communications, most MPI implementations still require the remote target to make 
MPI calls to ensure progress on such operations, consequently the operation 
cannot complete at the target without explicit processing in software and 
thus may cause arbitrarily long delays if the target process is busy computing 
outside the MPI stack. Traditional implementations to ensure asynchronous 
completion of operations have relied on a background thread that is 
dedicated to each MPI process. This approach, however, leads to wasted resources
because half of the cores on a node must be designated to background threads. 
Furthermore, this model forces multithreaded communication overhead on all MPI 
operations, which can be expensive. In this thesis, our third contribution is 
a \textit{process-based asynchronous progress model for MPI on many-core 
architectures utilizing dedicated helper cores}. This approach achieves better 
flexibility that user is allowed to define any arbitrary breakup of how many 
cores are assigned to the user processes and how many are assigned as helper 
cores without additional lock contention as that in thread-based design. Moreover, 
PMPI-based redirection helps separate from any MPI implementation so that various 
MPI implementations are able to be easily supported. 

\clearpage

